\documentclass[11pt, a4paper,spanish]{article}

\usepackage[utf8]{inputenc}
\usepackage{geometry}
\usepackage{graphicx}


\usepackage{hyperref}

\usepackage{float}

\newcommand\abs[1]{\left|#1\right|}


\title{Visualización de Datos MedioAmbientales de Castilla y León}
\author{Sergio García Prado}


\begin{document}

	\begin{titlepage}
		\centering
		{\scshape\LARGE Universidad de Valladolid \par}
		
		\vspace{1cm}
		{\scshape\Large Algoritmos y Computación\par}
		
		\vspace{1.5cm}
		{\huge\bfseries Job Shop Scheduling\par}
		
		\vspace{2cm}
		{\Large\itshape Sergio García Prado\par}

		\vfill
		Seguimiento del trabajo en: \par
		\href{https://github.com/garciparedes/Job-Shop-Scheduling-NP-Complete}{github.com/garciparedes/Job-Shop-Scheduling-NP-Complete}
		\vfill

		% Bottom of the page
		{\large \today\par}
	\end{titlepage}

	\newpage
		\tableofcontents
	\newpage
	
		\section{Introducción}
			
			\paragraph{}
			Job Shop Scheduling es un problema de optimización que se estudia en Ciencias de la Computación e Investigación de Operaciones. Ahora describiremos el problema y el objetivo que se pretende conseguir:
			
			\paragraph{}
			Supongamos que tenemos $n$ trabajos a realizar, los cuales denotamos como $J_{i}, i \in (1,n)$ cada uno de los cuales de distinta duración. También tenemos $m$ máquinas idénticas con las que realizar los trabajos.
			
			\paragraph{}
			El objetivo será tratar de minimizar al máximo posible el tiempo necesario para completar todas las tareas.
	
	
	
	\newpage

		\section{Bibliografía}

			\begin{itemize}
			
				\item
		
			\end{itemize}


\end{document}
